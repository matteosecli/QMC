%%%%%%%%%%%%%%%%%%%%%%%%%%%%%%%%%%%%%%%%%%%%%%%%%%%%%%%%%%%
%                                                         %
% PREFACE                                                 %
%                                                         %
% This file is part of a BSc Thesis Project. See the      %
% LICENSE file for more information about licensing.      %
%                                                         %
% Author:     Matteo Seclì <secli.matteo@gmail.com>       %
% A.Y.:       2014/2015                                   %
% URL:        https://github.com/matteosecli/QMC          %
%                                                         %
%%%%%%%%%%%%%%%%%%%%%%%%%%%%%%%%%%%%%%%%%%%%%%%%%%%%%%%%%%%

\chapter{Preface}
%Write something in the preface. Who inspired this project, why, 
%when....just think about something to write here.

Here I am, finally, writing this very last project as the ending of three years of Physics at the University of Trento.

My interest in computing and programming is quite an old story; I still remember that, as a child, I spent a lot of time learning the DOS 6 and how to program in Pascal and JavaScript. This interest has grown over the years and it eventually led me to discover new languages and new problems to solve. I've developed my own projects about the most different varieties of subjects: from basic cryptography to audio/video streams, to web-based platforms, to more scientific challenges with Matlab or Mathematica.

Until one year ago, I had no chance to blend this passion with the primary interest of my life: Physics. But then I've been offered a one-year exchange program at the University of Oslo, and I immediately chose ``Computational Physics'' as an optional class. The course exceeded my best expectations, mainly thanks to the professor, Morten Hjorth-Jensen, who did a great job in passing on his passion to his students. I've always spoken freely to him about those sides of Physics I like the most: the state-of-the-art mathematical formulations, the fundamental laws, and the recently-discovered quantum world (thanks to the course in ``Quantum Mechanics'' at the University of Oslo). In particular the last one had a great impact on my life, since I decided to follow the ``quantum path'' for my coming master studies.

Since he was well aware of my interests and I asked his guide for a more complex project, professor Morten Hjorth-Jensen proposed me a project tailored on my passions: a simulation of quantum dots, developed via Monte Carlo methods.




%%%%%%%%%%%%%%%%%%%%%%%%%%%%%%%%%%%%%%%%%%%%%%%%%%%%%%%%%%%
%                                                         %
% INTRODUCTION                                            %
%                                                         %
% This file is part of a BSc Thesis Project. See the      %
% LICENSE file for more information about licensing.      %
%                                                         %
% Author:     Matteo Seclì <secli.matteo@gmail.com>       %
% A.Y.:       2014/2015                                   %
% URL:        https://github.com/matteosecli/QMC          %
%                                                         %
%%%%%%%%%%%%%%%%%%%%%%%%%%%%%%%%%%%%%%%%%%%%%%%%%%%%%%%%%%%

\chapter{Introduction}
COMPLETE THIS INTRODUCTION WITH A MORE DETAILED EXPLANATION AND INDICATING WHAT THERE'S IN THE FIRST CHAPTER!!!

The aim of this project is to use the Variational Monte Carlo (VMC) 
method to evaluate the ground state energy, one-body densities, 
expectation values of the kinetic and potential energies and 
single-particle energies of quantum dots with $N = 2$ and $N = 6$ 
electrons, so-called \emph{closed shell systems}. We will begin with 
two electrons confined in a pure two-dimensional isotropic harmonic 
oscillator potential, with and without the repulsive interaction, developing
a simple trial wave-function and a basic program that runs our calculations.
Then we are going to improve the trial wave-function adding the so-called 
\emph{Jastrow factor}, and finally we are going to extend the entire class of
wave-functions and the program itself to a given number of electrons 
(obviously, for a closed-shell system). After that we are going to improve
our sampling, switching from a brute-force approach to the clever
\emph{importance sampling}. In addition, we will present a method that 
implements analytical derivatives for calculating both the energy and the
diffusive force of the importance sampling.			

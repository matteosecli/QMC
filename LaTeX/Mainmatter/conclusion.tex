%%%%%%%%%%%%%%%%%%%%%%%%%%%%%%%%%%%%%%%%%%%%%%%%%%%%%%%%%%%
%                                                         %
% CONCLUSION                                              %
%                                                         %
% This file is part of a BSc Thesis Project. See the      %
% LICENSE file for more information about licensing.      %
%                                                         %
% Author:     Matteo Seclì <secli.matteo@gmail.com>       %
% A.Y.:       2014/2015                                   %
% URL:        https://github.com/matteosecli/QMC          %
%                                                         %
%%%%%%%%%%%%%%%%%%%%%%%%%%%%%%%%%%%%%%%%%%%%%%%%%%%%%%%%%%%

\chapter{Conclusions and perspectives}
I hope that this was an interesting introduction about both quantum dots and the VMC method.

We explored the basic concepts about these devices, but this is just the tip of the iceberg. There are tons of resources about more advanced topics like improved potentials, how to deal with deformation, the effects of a magnetic field, the electronic structure of quantum wires and quantum rings, and so on.

There is an entire world also on the algorithms side. In this project we really showed the power of the VMC method that -- even if it doesn't give automatically the ground state energy of a system -- can be used to approximate it -- to excess -- with a considerable degree of precision. If we have no clues about the properties of the system we can use a brute force sampling, trying to manually set a step-length that gives about an acceptance ratio of $50\%$. But if we are smarter, we can use the information we have about the expected result to implement a more efficient sampling.

This program has been implemented in a highly object-oriented way, following the suggestions contained in \cite{Hoegberget2013}, and can also fully handle the 12-electrons case (look at the implementation of the orbitals in the code). However, it is still \emph{extremely slow} -- practically unusable. So, I'm planning to improve this program using MPI and GPU parallelization, that would give me some extra speed.

I'm also planning to extend this program in such a way that it can handle a custom number of variational parameters, and to improve the trial wave-function with some extra terms as suggested by professor Francesco Pederiva and/or using the Hartree-Fock method. This would require to abandon the analytical derivatives route and concentrate the efforts at improving the algorithm in terms of efficiency.

For me this project has been a springboard into the quantum computational physics world; I enjoyed so much doing this simulation that I've decided to go deeper into this subject and start to explore its full potential in advanced master courses. In particular I want to study improved methods like DMC and DFT, and acquire the theoretical knowledge necessary to fully understand these systems.

I'm also very curious about the world of quantum computation, that is pretty new and still in a dynamic development. It would be interesting to explore how the knowledge I acquired in this project -- once extended in advanced courses as I said -- could be used to develop new concepts and solutions in that field, and vice-versa. I'm planning to take two classes in quantum computation during my master studies, so I hope that I will be able to answer this question very soon -- or at least to start to.